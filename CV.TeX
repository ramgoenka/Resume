\documentclass[11pt,a4paper,sans]{moderncv}
\usepackage{xcolor}
\definecolor{mycolor}{RGB}{34,139,34} 
\moderncvstyle{classic}                            
%\moderncvcolor{blackblue} % CV color - options include: 'blue' (default), 'orange', 'green', 'red', 'purple', 'grey' and 'black'                            
\moderncvcolor{blue}

\usepackage[margin=0.75in]{geometry}
\setlength{\footskip}{149.60005pt}                 
\ifxetexorluatex
  \usepackage{fontspec}
  \usepackage{unicode-math}
  \defaultfontfeatures{Ligatures=TeX}
  \setmainfont{Latin Modern Roman}
  \setsansfont{Latin Modern Sans}
  \setmonofont{Latin Modern Mono}
  \setmathfont{Latin Modern Math}
\else
  \usepackage[utf8]{inputenc}
  \usepackage[T1]{fontenc}
  \usepackage{lmodern}
\fi
\def\CC{C\nolinebreak\hspace{-.05em}\raisebox{.4ex}{\tiny\bfseries ++}}
\usepackage[english]{babel}
\usepackage[unicode]{hyperref}
\usepackage{soul}
\usepackage{datetime2}

\name{Ram}{Goenka}
\title{Curriculum Vitae}
\phone[mobile]{+1 (217)~974~1713}        
\email{rag334@illinois.edu}         
\homepage{ramgoenka.github.io} 
\extrainfo{
    \faLinkedin\href{https://www.linkedin.com/in/ram-goenka/}{Linkedin} 
    }

\begin{document}
\makecvtitle

\section{Research Interests}
Statistical Learning, Markov Processes, Information Theory, Communication Theory, Digital and Statistical Signal Processing
\section{Education}
\cventry{Aug. 2025 -- May 2027}{M.S. Statistics, University of Pittsburgh}{}{}{}{
  %Thesis: \textit{Topic here}\\
  %Advisor: Name here\\
  %GPA: X.XX
}

\cventry{\emph{Aug. 2021 -- May 2025}}{B.S. Mathematics, B.S. Statistics}{University of Illinois Urbana-Champaign}{}{}{
  Minor in Computer Science\\
}
\section{Research Experience}
\cventry{\emph{Aug. 2023 -- Present}}{Undergraduate Research Assistant}{\href{https://www.ncsa.illinois.edu/}{\color{blue}\ul{National Center for Supercomputing Applications}}}{}{}{
  Mentor: Prof.\ Rebecca Lee Smith, University of Illinois Urbana-Champaign
  \begin{itemize}
    \item Transformed mathematical and statistical models into interactive RShiny applications to enhance vector control research, funded by the Center for Disease Control (CDC) and Midwest Center of Excellence for Vector Borne Disease (MCEVBD)
    \item Composed efficient code to process large datasets and create dynamic visualizations, including time series graphs and interactive geographical maps
    \item Implemented Generalized Additive Models (GAMs) to analyze data, calculating and displaying key inferential statistics to enhance data interpretation and regression analysis
  \end{itemize}
}

\cventry{\emph{Jun. 2024 -- Aug. 2024}}{Undergraduate Research Assistant}{\href{https://geometrynyc.wixsite.com/polymathreu}{\color{blue}\ul{Polymath Jr.}}}{}{}{
  Mentor: Prof.\ Alexandra Seceleanu, University of Nebraska-Lincoln
  \begin{itemize}
    \item Collaborated with fellow undergraduate researchers to study if (and how) combining two Macaulay posets in various ways (cartesian, wedge, diamond products) leads to another Macaulay poset
    \item Composed algorithms to analyze the additivity of posets and determine if a given poset is Macaulay or not. Implemented these algorithms in Macaulay2 language code for usage
    \item Compiled documentation on research findings and key theorems, as well as written code
  \end{itemize}
}

\cventry{\emph{Aug. 2023 -- Dec. 2023}}{Undergraduate Research Assistant}{\href{https://asrm.illinois.edu/illinois-risk-lab/}{\color{blue}\ul{Illinois Risk Lab}}}{}{}{
  Mentor: Prof.\ Runhuan Feng \& Dr.\ Peixin Liu, University of Illinois Urbana-Champaign
  \begin{itemize}
    \item Conducted research on the evolution, history, and current state of Decentralized Autonomous Organizations (DAOs)
    \item Compiled findings in a report synthesizing research findings and case studies to provide insightful perspectives on the development and future potential of DAOs
    \item Presented research findings to a panel of professors from the Department of Actuarial Science
  \end{itemize}
}

\section{Teaching Experience}

\cventry{\emph{Aug. 2022 -- May 2025}}{Undergraduate Teaching Assistant}{STAT 107: Data Science Discovery}{}{}{
  University of Illinois Urbana-Champaign
  \begin{itemize}
    \item Led Python labs ($\sim$ 30 students) aiding with statistical concepts, programming, and debugging
    \item Conducted office hours for students offering guidance on data science concepts, programming, statistical concepts, homework problems, labs, Python micro-projects and exam reviews
    \item Composed statistics and programming homework problems in concepts such as hypothesis testing, descriptive statistics, probability, linear regression, and programming in Python
  \end{itemize}
}
\cventry{\emph{May 2022 -- May 2023}}{Undergraduate Teaching Assistant}{CS 124: Intro. to Computer Science I}{}{}{
  University of Illinois Urbana-Champaign
  \begin{itemize}
    \item Guided students in computer science basics and the Java programming language through office hours and course forums. Hosted quiz-review sessions answering conceptual questions
    \item Refined course material ensuring correctness. Recorded homework walk-through's breaking down complex concepts
    \item Mentored eight first-time undergraduate TAs to familiarize them with the course interface and methodologies as well as expectations of being a course staff member
  \end{itemize}
}

\section{Professional Experience}

\cventry{\emph{May 2024 -- Aug. 2024}}{Data Analytics Intern}{Synchrony Financial}{}{}{
  \begin{itemize}
    \item Composed and optimized complex SQL queries to manipulate and aggregate datasets for over 2 million credit accounts for advanced analytics on the recovery strategy and collections team
    \item Developed predictive models and conducted statistical analyses using SAS, identifying trends and optimizing debt collection strategies. Utilized findings to identify \$20 million in potential gains
    \item Compiled findings in concise reports and presented data-driven strategies to senior leadership
  \end{itemize}
}

\cventry{\emph{May 2023 -- Aug. 2023}}{Software Engineering Intern}{COUNTRY Financial}{}{}{
  \begin{itemize}
    \item Refactored code for insurance processes and calculations on large datasets from SAS to Python using Pandas, achieving a 15\% improvement in performance and speed. Implemented unit tests using PyTest for validation
    \item Migrated and deployed on-premises Spring applications to Microsoft Azure Cloud, enhancing efficiency and performance. Documented the Azure deployment process for future company use
    \item Developed a proof of concept for an insurance-focused generative AI model using Azure OpenAI and LangChain, training it on relevant insurance concepts and the company database
  \end{itemize}
}

\cventry{\emph{Sept. 2022 -- Aug. 2023}}{Software Engineering Intern}{National Center for Supercomputing Applications}{}{}{
  \begin{itemize}
    \item Collaborated with the National Center for Atmospheric Research (NCAR) to develop a web interface for atmospheric chemistry simulations pertaining to aerosol particles
    \item Wrote Python code to develop time-series models for tracking aerosol particle concentrations utilizing atmospheric data from NetCDF files, and D3.js for the frontend to display the plots
    \item Improved the website backend to support larger file uploads and optimized frontend-to-backend efficiency for fast, accurate and refined plotting of data
  \end{itemize}
}

\section{Selected Coursework}
At UIUC:
\begin{itemize}
    \setlength\itemindent{2.5cm}
    \item STAT 431: Applied Bayesian Analysis
    \item STAT 432: Basics of Statistical Learning
    \item STAT 433: Stochastic Processes
    \item MATH 442: Intro. to Partial Differential Equations
    \item MATH 447: Real Variables
    \item CS 441: Applied Machine Learning
    \item CS 498DDU: End-to-End Data Science
\end{itemize}
%At Pitt
%\begin{itemize}
    %\setlength\itemindent{2.5cm}
%\end{itemize}

\end{document}
